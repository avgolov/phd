\chapter*{Выводы}						% Заголовок
\addcontentsline{toc}{chapter}{Выводы}	% Добавляем его в оглавление


Для моделирования газообмена в легких, была выполнена декомпозиция их структуры на проводящую зону (первые 4е поколения) и альвеолярные компоненты (8 компонент). Движение воздуха в проводящей зоне легких для каждой бронхиальной трубки описывается уравнением сохранения массы и импульса, которые дополняются "уравнением состояния". При численной реализации используется явная двухшаговая гибридная схема, соответствующая наиболее точной монотонной схеме первого порядка точности. В области стыковки трубок, имеющей дихотомическую структуру, решается система уравнений Бернулли и сохранения массы, которая дополняется условием совместности вдоль соответствующих характеристик. Движение воздуха и перенос дыхательных газов в переходной зоне описывается интегральной моделью для каждого альвеолярного объема. Численный расчет уравнений модели выполняется совместно с уравнениями проводящей зоны для конечных бронхиальных трубок с использованием условий равенства давлений и сохранения массы. 1D сетевая структура легких, используемая при расчетах в проводящей зоне, получена на основе обработки CT-данных пациента. 

Таким образом, совместная 0D-1D модель с одной стороны позволяет учесть индивидуальную специфику пациента, а с другой является экономичной с вычислительной точки зрения по сравнению, например, с подробными 3D моделями.     

Выполнена проверка адекватности 0D-1D модели дыхательной системы: 
\begin{itemize}
    \item 
    Получено хорошее совпадение альвеолярного потока и альвеолярного давления в нормальных условиях с литературными данными. 
    \item 
    Выполнен поиск оптимальной чистоты(обеспечивающей наилучший газообмен) ИВЛ. Полученное значение находится достаточно близко к экспериментально измеренной величине чистоты дыхания пациента в нормальном состоянии.
    \item
    Выполнено моделирование патологических рисунков дыхания: дыхание Биота и Чейна-Стокса. Сравнение с нормальным синусоидальным дыханием подтвердило, что при наличии патологии газообмен становится менее эффективным.
    \item
    Выполнено моделирование легких, умеренных и тяжелых приступов астмы. Определены параметры, соответствующие каждому типу приступа. Результаты качественно совпали с описанием из клинической литературы, к сожалению не удалось найти количественные значения, с которыми можно было бы провести сравнение.
\end{itemize}
 
Для расчета транспорта газов в кровеносной системе используется модель, основанная на разделении системы на пять крупных отделов, соответствующих артериям тканей, головного мозга и легких, системным и легочным венам. Модель учитывает связывание $O_{2}$ гемоглобином и механизм поддержания кислотно-щелочного баланса в крови. При физической нагрузке в модели учитываются механизмы производства и утилизации лактата (рассматривается переход от аэробного к анаэробному энергообмену), а также образование неметаболических излишек $CO_{2}$. Регуляция сердечного выброса и перераспределения кровотока между головным мозгом и тканями описывается эмпирической зависимостью от парциального давления $CO_{2}$ в системных артериях. Модель кровеносной системы и мышечного метаболизма дополняются усредненной моделью дыхательной системы (подробная 0D-1D модель для рассматриваемых практических задач является излишней). Регуляция параметров легочной вентиляции описывается эмпирической зависимостью от парциального давления $CO_{2}$, $O_{2}$ в системных артериях (периферический регулятор) и $CO_{2}$ в артериях головного мозга (центральный регулятор). Модель сводится к последовательному решению 4х жестких систем ОДУ, для численного расчета которых применяется A,L~-- устойчивый метод третьего порядка аппроксимации из семейства схем Обрешкова, а также итерационной процедуре совместного расчета газообмена с легкими. 

Данная комплексная модель описывает основные подсистемы и регуляторные механизмы, которые влияют на энергообмен во время физической нагрузки и являются критическими для достижения высокого спортивного результата. При этом, неизвестные параметры модели могут быть оценены по результатам тестов с возрастающей нагрузкой с помощью алгоритмов глобальной оптимизации.

Значение анаэробного порога(используется в модели мышечного метаболизма, а также имеет ключевое значение при оптимизации тренировочных нагрузок) определяется с помощью робастного алгоритма регрессии по нескольким физиологическим показателям. Данный алгоритм был протестирован и показал хорошее совпадение с оценками эксперта-физиолога.

Модель регуляции дыхания была протестирована на задаче моделирование минутной вентиляции легких в условиях гиперкапнии. Достигнута высокая степень согласованности между результатами численного моделирования и лабораторных исследований. Модель позволяет удовлетворительно воспроизвести установившееся значение минутной вентиляции. 

По результатам тестов с возрастающей нагрузкой для десяти спортсменов была выполнена идентификации параметров модели мышечного метаболизма. Для части параметров дисперсия (оценена методом бутстреппинга) получилась достаточна большой, а для других не превышала 5\%. Тем не менее модель показала хорошее совпадение с экспериментом.

\clearpage