\chapter*{Введение}							% Заголовок
\addcontentsline{toc}{chapter}{Введение}	% Добавляем его в оглавление

	%Обзор, введение в тему, обозначение места данной работы в мировых исследованиях и т.п.
\textbf{Актуальность темы исследования} 

Глобальный транспорт газов дыхательной и кровеносной системами является одним из основных процессов, связанных с жизнедеятельностью организма человека. Значительные и/или длительные отклонения от нормальных значений концентраций кислорода и углекислого газа в крови могут приводить к существенным патологическим изменениям, вызывающим необратимые последствия: недостаток кислорода (гипоксия и ишемические явления), изменение кислотно-щелочного баланса крови (ацидоз или алкалоз) и др. В условиях изменяющейся внешней среды и внутреннего состояния организма действие его регуляторных систем направлено на поддержание гомеостаза. 

Многие типичные изменения рисунка дыхания, такие как периодические и кластерные рисунки, связаны с конкретными заболеваниями и очень хорошо описаны в медицинской литературе. Кроме того, существует другая причина изменения картины дыхания: респираторная дисфункция. В этом случае имеют место хронические или повторяющиеся изменения в характере дыхания, которым нельзя дать конкретный медицинский диагноз, приводящие к жалобам пациентов с совершенно различными симптомами, такими как беспокойство, головокружения, усталость и т.д. В настоящее время не существует золотого стандарта для диагностики таких изменений рисунков дыхания вне их клинического описания. Более того, большинство специалистов по численному моделированию дыхательной системы в своих работах не уделяют особого внимания такого рода патологиям. Также для эффективного решения практических задач необходимо учитывать индивидуальную специфику каждого пациента, при этом возникает дилемма "точность модели/экономичность вычислений". Наибольшая точность может быть получена с помощью совместного решения трехмерных уравнений Навье-Стокса и уравнений деформируемого твердого тела по геометрии легких на основе данных компьютерной томографии(CT). Однако, для решения такой задачи обычно требуется вычислительный кластер либо длительное ожидание получения результата. Поэтому разработка моделей, которые с одной стороны максимально персонифицированы, и с другой стороны могут быть запущены на обычных компьютерах, является достаточно актуальной. 

Примером применения моделей транспорта дыхательных газов в медицине в условиях изменяющейся внешней среды является искусственная вентиляция легких(ИВЛ). При проведении хирургических операций под анестезией параметры ИВЛ контролируется анестезиологом, который выполняет корректировку параметров по отклонению параметров жизнедеятельности пациента. При такой схеме существует риск пере растяжения легких либо схлопывания альвеолярных объемов. В последнее время появляется большое количество работ, связанных с численным исследованием различных режимов ИВЛ. Одним из сдерживающих факторов в этом направлении является отсутствие достаточного объема медицинских данным в открытом доступе. Для данного направления также характерна проблема "точность модели/экономичность вычислений".   

Еще одной важной практической областью является моделирование глобального транспорта дыхательных газов при физической нагрузке. Одним из наиболее эффективных способов повышения спортивного результата является оптимизация тренировочного процесса. При этом важно индивидуальное дозирование уровня нагрузки и периодов отдыха. Дозировка достигаемого тренировочного воздействия производится тренером обычно опосредованно, на основе задания определенных значений "ключевых" параметров нагрузки: вида применяемых упражнений, их интенсивности и продолжительности, числа повторений упражнения и т.д. При этом предполагается, что между параметрами задаваемой нагрузки и изменениями тренируемой функции существуют определенные зависимости, на основе которых и становится возможным опосредованное управление тренировочным эффектом. Математические модели реакции организма на физическую нагрузку, с помощью которых можно рассчитывать тренировочный эффект, либо очень упрощенны и позволяют воспроизводить результаты только определенных протоколов нагрузки и не имеют большой прогностической способности, либо наоборот сложны(учитывают большое количество реакции), но очень локальны (подробно описывают процессы в определенной области организма) и не позволяют рассчитывать глобальный эффект. Поэтому критерием применимости моделей на практики является учет основных лимитирующих систем организма, позволяющий воспроизводить реакцию на различные протоколы нагрузки. Другой важной проблемой этой области является возможность идентификации модели по стандартным физиологическим тестам без применения сложного измерительного оборудования.         

\textbf{Цели и задачи:} 

Целью работы является разработка вычислительных методов расчета глобального транспорта дыхательных газов в организме человека при изменении условий внешней и внутренней среды.

Для достижения поставленной цели решаются следующие задачи:

\begin{itemize}
\item
Разработка вычислительного метода расчета альвеолярного газообмена с использованием 0D-1D математической модели на основе персонифицированных CT-данных пациента  
\item
Тестирование вычислительного метода применительно к поиску оптимальных параметров искусственной вентиляции легких
\item
Численное исследование изменений рисунков дыхания и эффективности альвеолярного газообмена при наличии различных патологий (периодическое и кластерное дыхание, астма)
\item
Разработка алгоритма определения аэробного и анаэробного порогов по нескольким физиологическим показателям
\item
Разработка комплексной модели глобального транспорта в организме при физической нагрузке
\item
Разработка методики идентификации параметров модели по экспериментальным данным тестов с возрастающей нагрузкой.
\end{itemize}

\textbf{Основные положения, выносимые на~защиту:}
\begin{enumerate}
 \item
 Вычислительный, совместный 0D-1D метод исследования альвеолярного газообмена с использованием CT-данных легких 
 \item
 Программный комплекс для проведения вычислительных персонифицированных экспериментов по исследованию рисунков дыхания при наличии патологий, оптимизации параметров искусственной вентиляции 
 \item
 Программный комплекс для определения аэробного и анаэробного порогов у спортсменов по результатам тестов с возрастающей нагрузкой с использованием нескольких физиологических показателей
 \item
 Вычислительная метод исследования газообмена у спортсменов при физической нагрузке на основе математической модели и результатов натурного эксперимента
   
\end{enumerate}

\textbf{Научная новизна:}
\begin{enumerate}
 \item 
Разработан вычислительный, совместный 0D-1D метод исследования альвеолярного газообмена, который позволяющий учитывать персонифицированную структуру легких и при этом не требует значительных вычислительных ресурсов.
 \item 
 Разработан численный алгоритм определения анаэробного порога, который позволяет использовать несколько физиологических показателей для сильно зашумленных данных нагрузочного тестирования
 \item
 Разработана комплексная модель газообмена у спортсменов при физической нагрузке, включающая в себя модели основных лимитирующих физические возможности систем: дыхательной, кровеносной, мышечного метаболизма, регуляции дыхания и кровообращения
 \item
 Разработана методика идентификации параметров модели по данным натурных экспериментов~--тестов с возрастающей нагрузкой
  
\end{enumerate}

\textbf{Научная и практическая значимость:} 

Подход из алгоритма стыковки 0D и 1D моделей легких может быть использован в  других, отличных от рассмотренных в работе, моделях легких.

Разработанный программный комплекс позволяет определять оптимальные параметры искусственной вентиляции на основе персональных CT-данных легких, а также исследовать изменения рисунка дыхания и эффективности альвеолярного газообмена при наличии различных патологий. 

Алгоритм расчета аэробного и анаэробного порогов, реализованный в виде программного комплекса, используется экспертами-физиологами для быстрой и эффективной обработки данных нагрузочного тестирования.

Составные части комплексной модели реакции организма на физическую нагрузку могут быть интегрированы с другими более подробными моделями (например с 1D моделью гемодинамики).

Вычислительная методика идентификации параметров мышечного метаболизма по результатам нагрузочного тестирования может быть использована спортивными тренерами при оптимизации тренировочного процесса.

\textbf{Степень достоверности и апробация результатов.} Основные результаты работы были представлены на следующих научных конференциях и семинарах:

\noindent
\begin{itemize}
  
  \item 7-я (2015 г.), 8-я (2016 г.), 9-я (2016 г.) конференции по математическим моделям и численным методам в биоматематике (ИВМ РАН, г. Москва, Россия); 
  
  \item конференция "Экспериментальная и компьютерная биомедицина" (УрФУ, г.Екатеринбург, 10-12 апреля 2016)
  
  \item 57-я (2014), 58-я (2015) научные конференции МФТИ "Современные проблемы фундаментальных и прикладных наук" (г. Долгопрудный, Московская область, Россия)
  
  \item 3-я (2015), 4-я (2016) научно-практическая конференция "Инновационные технологии в подготовке спортсменов" (Центр спортивных технологий Москомспорта, г. Москва)
  
  \item семинары "Математические методы и модели в задачах спорта" (Центр спортивных технологий Москомспорта, г. Москва, Россия, 24 сентября 2015, 14 июня 2016 )

\end{itemize}

Программный комплекс по определению аэробного и анаэробного порогов в настоящее время внедрен и используется специалистами ЦСТиСК Москомспорта.

Поданы заявки на регистрацию программ ЭВМ:
\begin{itemize}
    \item "Программа расчета глобального транспорта газов в организме человека"
    \item "Программа расчета аэробного и анаэробного порогов спортсменов по данным нагрузочного тестирования"
\end{itemize}

\textbf{Публикации.} 
Основные результаты по теме диссертации опубликованы в 8 статьях и сборниках трудов конференций~\cite{GolovComp2017, GolovCmodel2017, GolovIt2017,GolovSp2015,GolovSp2016,TimmeSp2016,GolovEkb2016,Simakov2015}, из которых 2 изданы в журналах, рекомендованных ВАК \cite{GolovIt2017, GolovCmodel2017}, и 2 присутствуют в международных базе цитирования Web of Science \cite{GolovComp2017, GolovCmodel2017}.

\textbf{Личный вклад.}
В работах \cite{GolovComp2017,Simakov2015} автором предложен численный алгоритм стыковки 0D и 1D моделей легких. Выполнена проверка адекватности модели на тестовых задачах. Найдены оптимальные параметры искусственной вентиляции для пациента по CT-данным. Проведено исследование альвеолярного газообмена при наличии патологий: периодическое и кластерное дыхание, астма.

В работе \cite{GolovCmodel2017} автором разработана модель биохимии крови, регуляции дыхания и кровообращения. Выполнена проверка корректности модели в сравнении с экспериментальными данными по изменению минутной вентиляции в условиях гипоксии и гиперкапнии.

В работах \cite{GolovIt2017,GolovSp2016, TimmeSp2016} автором предложен алгоритм определения аэробного и анаэробного порогов по нескольким физиологическим показателям. Выполнено сравнение результатов работы алгоритма с оценками эксперта-физиолога.

В работах \cite{GolovSp2015,GolovEkb2016} автором разработана модель мышечного метаболизма, которая была интегрирована в комплексную модель реакции организма на физическую нагрузку. Было выполнено моделирование и сравнение с экспериментальными данными тестов с возрастающей нагрузкой.

\textbf{Объем и структура работы.} Диссертация состоит из~введения, четырех глав, заключения и~четырёх приложений. Полный объем диссертации составляет 145 страницу с~67~рисунками и~20~таблицами. Список литературы содержит 137~наименований.

\clearpage