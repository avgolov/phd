\clearpage
\phantomsection
\addcontentsline{toc}{chapter}{\listfigurename}
\listoffigures									% Список изображений
\newpage

%%% Список таблиц %%%
% (ГОСТ Р 7.0.11-2011, 5.3.10)
\clearpage
\phantomsection
\addcontentsline{toc}{chapter}{\listtablename}
\listoftables									% Список таблиц
\newpage

\clearpage
%\phantomsection
\chapter*{Список сокращений и терминов} \label{AbrList}
\addcontentsline{toc}{chapter}{Список сокращений и терминов}

\textbf{CT} --- компьютерная томография

\textbf{CFD} --- вычислительная гидродинамика

\textbf{FSI} --- fluid-structure interaction

\textbf{DPG} --- дифосфонооксипропановая кислота

\textbf{LA} --- лактат

\textbf{АэП} --- аэробный порог. Это уровень нагрузки, при которой образование лактата в скелетной мышце превышает его распад, поэтому лактат начинает постепенно накапливаться в общей системе циркуляции.

\textbf{ПАНО} --- порог анаэробного обмена. Нагрузка, при которой лактат начинает накапливаться с очень большой скоростью.

\textbf{ДИ} --- доверительный интервал

\textbf{VCO2} --- выделение углекислого газа

\textbf{RER} --- отношение потребления кислорода к выделению углекислого газа

\textbf{VE} --- минутная вентиляция легких

\textbf{ExсCO2} --- дополнительное выделение углекислого газа по сравнению с состоянием покоя

\textbf{КТЭ} --- кумулятивный тренировочный эффект 

\textbf{Метода бутстреппинга} --- практический компьютерный метод исследования распределения статистик вероятностных распределений, основанный на многократной генерации выборок методом Монте-Карло на базе имеющейся выборки. Позволяет просто и быстро оценивать самые разные статистики (доверительные интервалы, дисперсию, корреляцию и так далее) для сложных моделей.

\newpage