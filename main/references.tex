\clearpage
\phantomsection
\addcontentsline{toc}{chapter}{\bibname}	% Добавляем список литературы в оглавление

\begin{thebibliography}{}
    
    \bibitem{GolovComp2017}{\it \textbf{Golov A.}, Simakov S. , Soe Y.N., Pryamonosov R., Mynbaev O., Kholodov A.} Multiscale CT-Based Computational Modeling of Alveolar Gas Exchange during Artificial Lung Ventilation, Cluster (Biot) and Periodic (Cheyne-Stokes) Breathings and Bronchial Asthma Attack // MDPI Computation. ~---2017;~---Vol.~5(1), 11.~---P.1-18
    
    \bibitem{GolovCmodel2017}{\it \textbf{Голов А.В.}, Симаков С.С.} Математическая модель регуляции легочной вентиляции при гипоксии и гиперкапнии // Компьютерные исследования и моделирование. ~---2017;~---Vol.~2(9).~---С.297–310
    
    \bibitem{GolovIt2017}{\it \textbf{Голов А.В.}, Тимме Е.А., Козлов А.В.} Алгоритм автоматизированной оценки параметров работоспособности человека при выполнении нагрузочных тестов // Информационные технологии ~---2017;~---Vol.~4(23).~---С.309–314

    \bibitem{GolovSp2015}{\it \textbf{Голов А.В.}, Тимме Е. А., Симаков С. С., Холодов А. С.} Математическое моделирование альвеолярного газообмена при проведении нагрузочных тестов // материалы 3-ей научно-практическая конференции «Инновационные технологии в подготовке спортсменов» ~---2015;~---Vol.~1.~---С.25–28
    
    \bibitem{GolovSp2016}{\it \textbf{Голов А.В.}, Тимме Е.А., Козлов А.В.} Автоматизированная оценка аэробного и анаэробного порогов по результатам нагрузочного тестирования // материалы 4-ой научно-практической конференция «Инновационные технологии в подготовке спортсменов» ~---2016;~---Vol.~1.~---С.415-417
    
    \bibitem{TimmeSp2016}{\it Тимме Е.А., \textbf{Голов А.В.}} Математическая модель, алгоритм и программный модуль оценки параметров кинетики потребления кислорода при ступенчатом нагрузочном тесте // материалы Всероссийской научно-практической интернет-конференции «Актуальные проблемы биохимии и биоэнергетики спорта 21 века» ~---2016;~---Vol.~1.~---С.102-106

    \bibitem{GolovEkb2016}{\it \textbf{Golov A.V.}, Simakov S.S., Timme E.A.} Mathematical modeling of alveolar ventilation and gas exchange during treadmill stress tests // материалы российской конференции с международным участием «Экспериментальная и компьютерная биомедицина» ~---2016;~---Vol.~1.~---С.32
    
    \bibitem{Simakov2015}{\it Simakov S.S.,  \textbf{Golov A.V.}, Soe Y.N.} Mathematical modeling of alveolar ventilation and gas exchange during treadmill stress tests // Mathematical modeling of cardiovascular and  respiratory systems of human organism ~---2015;~---Vol.~2.~---P.42

	\bibitem{volkov2013}
	{\it Бреслав~И.\,С., Волков~Н.\,И., Тамбовцева~Р.\,В.} Дыхание и мышечная активность в спорте. ~---М: Советский спорт, 2013.
			
	\bibitem{Dyachenko1985}
	{\it Дьяченко~А.\,И., Шабельников~В.\,Г. и др.} Математические модели действия гравитации на функции легких. ~---М: Наука, 1985.
	
	\bibitem{Dyachenko1986} {\it Дьяченко~А.\,И.}, Исследование однокомпонентной модели механики легких.~// Медицинская биомеханика.---Рига, 1986.---Т.1.---C.147-152.
		
	\bibitem{matyushev14} {\it Матюшев~Т.\,В., Дворников~М.\,В., Богомолов~А.\,В., Кукушкин~Ю.\,А., Поляков~А.\,В.,}  Математическое моделирование динамики показателей газообмена человека в условиях гипоксии // Математическое моделирование.---2014.---Т.26.---N4.---С.51-64.
		
	\bibitem{Simakov08a} {\it Симаков~C.\,С., Холодов~А.\,С.,}  Численный анализ воздействия акустических возмущений на функцию легких и гемодинамику малого круга кровообращения // В сб. Медицина в зеркале информатики, М.: Наука, 2008, С.124-170.
			
	\bibitem{kholodov2001}
	{\it Холодов~А.\,С.} Некоторые динамические модели внешнего дыхания и кровообращения с учетом их связности и переноса веществ. // в сб. Компьютерные модели и прогресс медицины~---М: Наука, 2001, С.~127-163.
				
	\bibitem{bental2013} {\it Ben-Tal~A., Tawhai~M.\,H.} Integrative approaches for modeling regulation and function of the respiratory system.~//WIREs system biology and medicine. ~---2013.~---Vol.~5.~--- P.~687-699.
		
	\bibitem{cheng2014} {\it Cheng~L., Albanese~A., Ursino~M., Nicolas~W.\,C.} An integrated mathematical model of the human cardiopulmonary system: model validation under hypercapnia and hypoxia.~//American Journal of Physiology --- Heart and Circulatory Physiology. ~---2016.~---Vol.~310.~--- PP.~922-937.

	\bibitem{duffin2000} {\it Duffin~L., Mohan~R.\,M., Vasiliou~P., Stephenson~R., Mahamed~S.} A model of the chemoreflex control of breathing in humans: model parameters measurement.~//Respiration physiology. ~---2000.~---Vol.~120.~--- P.~13-26.

	\bibitem{gibson1970} {\it Gibson~Q.\,H.} The reaction of oxygen with hemoglobin and the kinetic basis of the effect of salt on binding of oxygen.~//The Journal of Biological Chemistry.~---1970.~---Vol.~245.~--- P.~3285-3288.
	
	\bibitem{guyton2000} {\it Guyton~A.\,C., Hall~J.\,E.} Textbook of Medical Physiology, 10 ed.,~---Philadelphia:  W.B. Saunders Company, 2000
	
	\bibitem{clark2002} {\it Lu~K., Clark Jr.,~W., Ghorbel~F.\,H., Ware~D.\,L., Zwischenbrger~J.\,B., Bidani~A.} Whole-body gas exchange in human predicted by a cardiopulmonary model.~//Cardiovascular Engineering: An International Journal.~---2002.~---Vol.~3, No.~1.~--- P.~1-19.
	

	\bibitem{mynbaev2002} {\it Mynbaev~O.\,A., Molinas~C.\,R., Adamyan~L.\,V., Bernard~V., Koninckx~Ph.\,R.}
	Pathogenesis CO2 pneumoperitoneum-induced metabolic hypoxemia in a rabbit model.~// The Journal of the American Association of Gynecologic Laparoscopists.~---2002.~---Vol.~9, No.~3.~--- P.~306-314.
	
	
	\bibitem{reynolds1972} {\it Reynolds~W.\,J., Milhorn~H.\,T., Holloman~G.\,H.} Transient ventilatory response to graded hypercapnia in man.~//Journal of applied physiology. ~---1972.~---Vol.~33, No.~1.~--- P.~47-54.
	
	\bibitem{reynolds1973} {\it Reynolds~W.\,J., Milhorn~H.\,T., Holloman~G.\,H.} Transient ventilatory response to hipoxia with and without controlled alveolar \(PCO_{2}\).~//Journal of applied physiology. ~---1973.~---Vol.~35, No.~2.~--- P.~187-196.
	
	\bibitem{schmidt} {\it Schmidt~R.\,F., Thews~G,} Human Physiology, 2nd ed.,~---Berlin:  Springer-Verlag, 1989
	
	\bibitem{Simakov08b} {\it Simakov~S.\,S., Kholodov~A.\,S.,} Computational study of oxygen concentration in human 	blood under low frequency disturbances // Mathematical models and computer simulations.~---2008.~---Vol.~1, No. 2.~---P.283-295.

	\bibitem{Simakov16} {\it Simakov~S.\,S.,  Roubliova~X.\,I., Ivanov~A.\,A., Kaptaeva~A.\,K., Mazhitova~M.\,I., Mynbaev~O.\,A.,} Respiratory acidosis in obese gynecological patients undergoing laparoscopic surgery independently of the type of ventilation.~// Acta Anaesthesiologica Taiwanica.~---2016.
	
	\bibitem{topor2004} {\it Topor~Z.\,L., Pawlicki~M., Remmers~J.\,E.} A computational model of the human respiratory control system: Responses to hypoxia and hypercapnia.~//Annals of Biomedical Engineering. ~---2004.~---Vol.~32, No.~11.~--- P.~1530-1545.
		
	\bibitem{wolf2007} {\it Wolf~M.\,B., Garner~R.\,P.} A mathematical model of human respiration at altitude~// Annals of Biomedical Engineering. ~---2007.~---Vol.~35, No.~11.~--- P.~2003-2022.
	
	\bibitem{prampero1999} {\it di Prampero~P\,E., Ferretti~G.} The energetics of anaerobic muscle metabolism: a reappraisal of older and recent concepts. ~//Respiratory Physiology. ~---1999.~Vol~118.~---P.~103-115. 
	
    \bibitem{prampero1981} {\it di Prampero~P} Energetics of muscular exercise. ~//Rev Physiol Biochem Pharmacol. ~---1981.~Vol~89.~---P.~144-222. 
    
    \bibitem{duke2015} {\it Dukes~H.\,H.} Dukes' Physiology of domestic animals, 13 ed.,~---Oxford: Wiley Blackwell, 2015
    
    \bibitem{maughan1997} {\it Maughan~R., Gleeson~M., Greenhaff~P.\,L.} Biochemistry of exercise and training, 1 ed.,~---Oxford: Oxford University Press, 1997
    
    \bibitem{Fakuba1993} {\it Fakuba~Y., Walsh~M.\,L., Cameron~R.\,H., Morton~R.\,H., Kenny~C.\,T., Banister~E.\,W.} Lactate modeling and its application to endurance training // J. therm. Biol.~---1993.~---Vol.~18., No. 5/6.~---P.617-622.
    
    \bibitem{Thomas2012} {\it Thomas~C., Bernard~O., Enea~C., Jalab~C., Hanon~C.} Metabolic and respiratory adaptations during intense exercise following long-sprint training of short duration // Eur J Appl Physiol.~---2012.~---Vol.~112.~---P.667–675.
    
    \bibitem{Freund1990} {\it Freund~H., Oyono-Enguelle~S., Heitz~A., Ott~C., Marbach~J., Gartner~M., Pape~A. } Comparative Lactate Kinetics Alter Short and
Prolonged Submaximal Exercise // Int. J. Sports Med~---1990.~---Vol.~11.~---P.284–288.

    \bibitem{Francaux1989} {\it Francaux~M.\,A.,  Jacqmin~P.\,A., Sturbois~X.\,G., } Simple kinetic model for the study of lactate metabolic adaptation to exercise in sportsmen routine evaluation // Archives Internationales de Physiologie et de Biochimie~---1989.~---Vol.~97.~---P.235–245.
    
    \bibitem{Beneke2005} {\it Beneke~R., Hutler~M., Jung~M., Leithauser~R.\,M.} Modeling the blood lactate kinetics at maximal short-term exercise conditions in children, adolescents and adults // J Appl Physiol~---2005.~---Vol.~99.~---P.499-504.
    
    \bibitem{Pedley1970}{\it Pedley T, Schroter R, Sudlow M.} The prediction of pressure drop and variation of resistance within the human bronchial airways // Respiration Physiology ~---1970;~---Vol.~9.~---P.387–405.
    
    \bibitem{Reynolds1979}{\it Reynolds D, Lee J.} Modeling study of the pressure-flow relationship of the bronchial tree // Federation
Proceedings ~---1979;~---Vol.~38.~---P.1444.

    \bibitem{Lambert1982}{\it Lambert R,Wilson T, Hyatt R, Rodarte J.} A computational model for expiratory flow // Journal of Applied Physiology ~---1982;~---Vol.~52.~---P.44-56.
    
    \bibitem{Ertbruggen2005}{\it van Ertbruggen C, Hirsch C, Paiva M.} Anatomically based three-dimensional model of airways to simulate flow and particle transport using computational fluid dynamics // Journal of Applied Physiology ~---2005;~---Vol.~98(3).~---P.970-980.
    
    \bibitem{Bates2009} {\it Bates~J.} Lung Mechanics: An Inverse Modeling Approach,~---New York:  Cambridge University Press, 2009
    
    \bibitem{Suki1993}{\it Suki B, Habib RH, Jackson AC.} Wave propagation, input impedance, and wall mechanics of the calf trachea from 16
to 1,600 hz // American Journal of Physiology ~---1993;~---Vol.~75.~---P.2755–2766.
    
    \bibitem{Lutchen1996}{\it Lutchen KR, Hantos Z, Petak F, Adamicza A, Suki B.} Airway inhomogeneities contribute to apparent lung tissue
mechanics during constriction // American Journal of Physiology ~---1996;~---Vol.~80.~---P.1841–1849.

    \bibitem{Gillis1999}{\it Gillis HL, Lutchen KR.} How heterogeneous bronchoconstriction affects ventilation distribution in human lungs: a
morphometric model // Annals of Biomedical Engineering ~---1999;~---Vol.~27.~---P.14-22.

    \bibitem{Nucci2002}{\it Nucci G, Tessarin S, Cobelli C.} A morphometric model of lung mechanics for time-domain analyses of alveolar
pressures during mechanical ventilation // Annals of Biomedical Engineering ~---2002;~---Vol.~30.~---P.537–545.

    \bibitem{Horsfield1971}{\it Horsfield K, Dart G, Olson D, Filley G, Cumming G.} Models of the human bronchial tree // Journal of Applied
Physiology ~---1971;~---Vol.~31(2).~---P.207.

    \bibitem{Weibel1963}{\it Weibel ER} Morphometry of the Human Lung, ~---Berlin-Göttingen-Heidelberg: Springer, 1963
Physiology ~---1971;~---Vol.~31(2).~---P.207.

    \bibitem{Tawhai2000}{\it Tawhai MH, Pullan AH, Hunter PJ.} Generation of an anatomically based three-dimensional model of the conducting
airways // Annals of Biomedical Engineering ~---2000;~---Vol.~28.~---P.93–802.

    \bibitem{Tawhai2010}{\it Tawhai M, Lin C.} Image-based modeling of lung structure and function // Journal of Magnetic Resonance Imaging ~---2010;~---Vol.~32(6).~---P.1421–1431.
    
    \bibitem{Tawhai2006}{\it Tawhai M, Nash M, Hoffman E.} An imaging-based computational approach to model ventilation distribution and
soft-tissue deformation in the ovine lung // Academic Radiology ~---2006;~---Vol.~13(1).~---P.113.

    \bibitem{Lin2009}{\it Lin C, Tawhai M, Mclennan G, Hoffman E.} Computational fluid dynamics: multiscale simulation of gas
flow in subject-specific models of the human lung // IEEE Engineering in Medicine and Biology ~---2009;~---Vol.~28(3).~---P.25-33.

    \bibitem{Denny2000}{\it Denny E, Schroter RC.} Viscoelastic behavior of a lung alveolar duct model // Journal of Biomechanical Engineering ~---2000;~---Vol.~112.~---P.143–151.
    
     \bibitem{Zhang2002}{\it Zhang Z, Kleinstreuer C.} Transient airflow structures and particle transport in a sequentially branching lung airway
model // Physics of Fluids ~---2002;~---Vol.~14(2).~---P.862–880.

    \bibitem{Liu2002}{\it Liu Y, So RMC, Zhang CH.} Modeling the bifurcating flow in a human lung airway // Journal of Biomechanics ~---2002;~---Vol.~35.~---P.465–473.
    
    \bibitem{Green2004}{\it Green AS.} Modelling of peak-flow wall shear stress in major airways of the lung // Journal of Biomechanics ~---2004;~---Vol.~37(5).~---P.661–667.
    
    \bibitem{Zhang2004}{\it Zhang Z, Kleinstreuer C.} Airflow structures and nano-particle deposition in a human upper airway model // Journal of
Computational Physics ~---2004;~---Vol.~198.~---P.178-210.

    \bibitem{Lin2007}{\it Lin CL, Tawhai MH, McLennan G, Hoffman EA.} Characteristics of the turbulent laryngeal jet and its effect on
airflow in the human intra-thoracic airways // Respiratory Physiology \& Neurobiology ~---2007;~---Vol.~157.~---P.295–309.

    \bibitem{LDe2008}{\it LDe Backer J, Vos W, Gorle C, Germonpré P, Partoens B, Wuyts F, Parizel P, De Backer W.} Flow analyses in the
lower airways: patient-specific model and boundary conditions // Medical Engineering \& Physics ~---2008;~---Vol.~30(7).~---P.872.

    \bibitem{Wall2008}{\it Wall WA, Rabczuk T.} Fluid structure interaction in lower airways of CT-based lung geometries // International Journal
for Numerical Methods in Fluids ~---2008;~---Vol.~57.~---P.653–675.

    \bibitem{Ma2006}{\it Ma B, Lutchen KR} An anatomically based hybrid computational model of the human lung and its application to low
frequency oscillatory mechanics // Annals of Biomedical Engineering ~---2006;~---Vol.~14.~---P.1691–1704.

    \bibitem{Comerford2010}{\it Comerford A, Förster C, Wall WA} Structured tree impedance outflow boundary conditions for 3D lung simulations// Journal of Biomechanical Engineering ~---2010;~---Vol.~132(8).~---P.1–10.
    
    \bibitem{Yoshihara2013}{\it Yoshihara L, Ismail M, Wall W.} Bridging Scales in Respiratory Mechanics in: Computer Models in Biomechanics, ~---Berlin:  Springer, 2013; ~---P.395–407.
    
    \bibitem{Yin2010}{\it Yin Y, Choi J, Hoffman E, Tawhai M, Lin C.} Simulation of pulmonary air flow with a subject-specific boundary
condition// Journal of Biomechanics ~---2010;~---Vol.~43(11).~---P.2159–2163.

    \bibitem{Quarteroni2003}{\it Quarteroni A, Veneziani A.} Analysis of a geometrical multiscale model based on the coupling of ODE and PDE for
blood flow simulations// Multiscale Modeling and Simulation ~---2003;~---Vol.~1(2).~---P.173–195.

    \bibitem{Formaggia2000}{\it Formaggia L, Gerbeau JF, Nobile F, Quarteroni A.} On the coupling of 3D and 1D Navier-Stokes equation
for flow problems in compliant vessels.// Computer Methods in Applied Mechanics and Engineering ~---2000;~---Vol.~191.~---P.561–582.

    \bibitem{Fernandez2005}{\it Fernandez MA, Milsic V, Quarteroni A.} Analysis of a geometrical multiscale blood flow model based on the coupling
of ODEs and hyperbolic PDEs// Multiscale Modeling and Simulation ~---2005;~---Vol.~4.~---P.215–236.

    \bibitem{Vito2003}{\it Vito R. P., Dixon S. A. } Blood vessel constitutive models-1995-2002 // Annual Review of Biomedical Engineering ~---2003;~---Vol.~5(1).~---P.413–439.
    
    \bibitem{Humphrey2003}{\it Humphrey J.D.} Continuum biomechanics of soft biological tissues  // Proceedings of the Royal Society if Lonon Series A- Mathematical Physical and Engineering Sciences. ~---2003;~---Vol.~459(2029).~---P.3-46.
    
    \bibitem{Chen2013}{\it Chen H., Luo T., Zhao X., Lu X., Huo Y., Kassab G. S.} Microstructural constitutive model of active coronary media  // Biomaterials ~---2013;~---Vol.~34(31).~---P.7575–7583.
    
    \bibitem{Christensen1996}{\it Christensen TG, Draeby C.} Modelling the respiratory systems. Master’s Thesis, 1996. (Available from: http://milne.ruc.dk/ImfufaTekster/pdf/318.pdf), IMFUFA tekster 318 [Accessed on 12/12/2012]).
    
    \bibitem{Siggaard1984}{\it Siggaard-Andersen O, Wimberley PD, Gthgen I, Siggaard-Andersen M.} A mathematical model of the hemoglobinoxygen
dissociation curve of human blood and of the oxygen partial pressure as a function of temperature  // Clinical Chemistry ~---1984;~---Vol.~30(10).~---P.1646–1651.

    \bibitem{Christensen2004}{\it Christensen TG, Draeby C.} Respiration. In Applied Mathematical Models in Human Physiology., ~---Philadelphia:  SIAM, 2004; ~---P.149–150.

    \bibitem{Chiari1994}{\it Chiari L, Avanzolini G, Grandi F, Gnudi G.} A simple model of the chemical regulation of acid-base balance in
blood.  // Engineering in Medicine and Biology Society. Engineering Advances: New Opportunities for Biomedical
Engineers. Proceedings of the 16th Annual International Conference of the IEEE ~---1994;~---Vol.~2.~---P.1025–1026.

    \bibitem{GuytonColeman1972}{\it Guyton AC, Coleman TG, Granger HJ.} Circulation: overall regulation  // Annual Review of Physiology ~---1972;~---Vol.~34.~---P.13–44.
    
    \bibitem{Grodins1967}{\it Grodins FS, Buell J, Bart AJ} Mathematical analysis and digital simulation of the respiratory control system  // Journal
of Applied Physiology ~---1967;~---Vol.~22(2).~---P.260-276.

    \bibitem{Gray1946}{\it Gray JS.} The multiple factor theory of the control of respiratory ventilation  // Science ~---1946;~---Vol.~103(2687).~---PP.739–744.
    
    \bibitem{Reyes2014}{\it de los Reyes V AA, Jung E, Kappel F.} Stabilizing control for a pulsatile cardiovascular mathematical model  // Bulletin
of Mathematica Biology ~---2014;~---Vol.~76.~---P.1306–1332.

    \bibitem{Mauro1998}{\it Ursino M.} Interaction between carotid baroregulation and the pulsating heart: a mathematical model // American
Journal of Physiology - Heart and Circulatory Physiology ~---1998;~---Vol.~275(5).~---P.1733–1747.

    \bibitem{Ursino1994}{\it Ursino M, Antonucci M, Belardinelli E.} Role of active changes in venous capacity by the carotid baroreflex:
analysis with a mathematical model // American Journal of Physiology - Heart and Circulatory Physiology ~---1994;~---Vol.~267(6).~---P.2531–2546.

    \bibitem{Batzel2007}{\it Batzel JJ, Kappel F, Schneditz D, Tran HT} Cardiovascular and Respiratory Systems: Modeling, Analysis, and
Control.., ~---Philadelphia:  SIAM, 2007.

   \bibitem{Ottesen2004}{\it Ottesen JT, Danielsen M.} A cardiovascular model. In Applied Mathematical Models in Human Physiology, ~---Philadelphia:  SIAM, 2004., PP.149-150
   
   \bibitem{Chiari1997}{\it Chiari L, Avanzolini G, Ursino M.} A comprehensive simulator of the human respiratory system: validation with
experimental and simulated data // Annals of Biomedical Engineering ~---1997;~---Vol.~25.~---P.985–999.

    \bibitem{Magosso2001}{\it Magosso E, Ursino M.} A mathematical model of CO2 effect on cardiovascular regulation // American Journal of
Physiology - Heart and Circulatory Physiology ~---2001;~---Vol.~281(5).~---P.2036–2052.

    \bibitem{Olufsen2013}{\it Olufsen MS, Ottesen JT} A practical approach to parameter estimation applied to model predicting heart rate
regulation // Journal of Mathematical Biology ~---2013;~---Vol.~67.~---P.39-68.

    \bibitem{Batzel2010}{\it Batzel J, Bachar M.} Modeling the cardiovascular-respiratory control system: data, model analysis, and parameter
estimation // Acta Biotheoretica ~---2010;~---Vol.~58.~---P.369–380.

    \bibitem{Fink2008}{\it Fink M, Batzel J, Tran H.} A respiratory system model: parameter estimation and sensitivity analysis // Cardiovascular
Engineering ~---2008;~---Vol.~8(2).~---P.120-134.

    \bibitem{Kappel2006}{\it Kappel F, Batzel J.} Sensitivity analysis of a model of the cardiovascular system // Proceedings of the 28th Annual
International Conference of the IEEE ~---2006;~---Vol.~1.~---P.359–362.

    \bibitem{Batzel2011}{\it Batzel J, Ellwein L, Olufsen M} Modeling cardio-respiratory system response to inhaled CO2 in patients with
congestive heart failure. // Proceedings of the 33rd Annual International Conference of the IEEE ~---2011;~---Vol.~58.~---P.2418–2421.

    \bibitem{Khoo2008}{\it Khoo M.} Modeling of autonomic control in sleep-disordered breathing // Cardiovascular Engineering ~---2008;~---Vol.~8(1).~---P.30–41.
    
    \bibitem{Ottesen2014}{\it Ottesen JT, Mehlsen J, Olufsen MS.} Structural correlation method for model reduction and practical estimation of
patient specific parameters illustrated on heart rate regulation // Mathematical Biosciences ~---2014;~---Vol.~257.~---P.50–59.

    \bibitem{Ellwein2013}{\it Ellwein L, Pope S, Xie A, Batzel J, Kelley C, Olufsen M.} Patient-specific modeling of cardiovascular and respiratory
dynamics during hypercapnia // Mathematical Biosciences ~---2013;~---Vol.~241.~---P.256-274.

    \bibitem{Williams2013}{\it Williams MD,Wind-Willassen O,Wright A, Mehlsen J, Ottesen J, Olufsen M.} Patient-specific modelling of head-up
tilt // Mathematical Medicine and Biology ~---2013;~---Vol.~31(4).~---P.365–392.

    \bibitem{Ursino2001}{\it Ursino M, Magosso E, Avanzolini G.} An integrated model of the human ventilatory control system: the response to
hypercapnia // Clinical Physiology ~---2001;~---Vol.~21(4).~---P.447–464.

    \bibitem{Hardy1982}{\it Hardy H, Collins R, Calvert R.} A digital computer model of the human circulatory system // Medical and Biological
Engineering and Computing ~---1982;~---Vol.~20(5).~---PP.550–564.

    \bibitem{Lu2001}{\it Lu K, Clark JW, Ghorbel FH, Ware DL, Bidani A.} A human cardiopulmonary system model applied to the analysis
of the valsalva maneuver // American Journal of Physiology - Heart and Circulatory Physiology ~---2001;~---Vol.~281(6).~---P.2661–2679.

    \bibitem{Hemalatha2010}{\it Hemalatha K, Suganthi L, Manivannan M.} Hybrid cardiopulmonary model for analysis of valsalva maneuver with
radial artery pulse // Annals of Biomedical Engineering ~---2010;~---Vol.~38(10).~---P.3151–3161.

    \bibitem{Blanco2016}{\it Trenhago P.R. ,Fernandes L.G.
Müller L.O., Blanco P.J., Feijóo R.A.} An integrated mathematical model of the cardiovascular and
respiratory systems // Int. J. Numer. Meth. Biomed. Engng ~---2016;~---Vol.~1.~---P.1–25.

    \bibitem{Simakov2009}{\it  Simakov S.S.; Kholodov A.S.} Computational study of oxygen concentration in human blood under low frequency disturbances // Mathematical models and computer simulations ~---2009;~---Vol.~1.~---P.283-295.
    
    \bibitem{Kholodov2006}{\it  Kholodov A.S., Simakov S.S. et.al}  Computational models on graphs for nonlinear hyperbolic and parabolic system of equations // Proceedings of III European Conference on Computational Mechanics ~---2006;~---Vol.~1.~---P.43.
    
    \bibitem{Vassilevski2011}{\it   Vassilevski Y.; Simakov S.; Dobroserdova T.; Salamatova V.}  Numerical issues of modelling blood flow in networks of vessels with pathologies. Russ. // J Numer Anal Math Mod ~---2011;~---Vol.~26.~---PP.605-622.
    
    \bibitem{Bessonov2016}{\it Bessonov N.; Sequeira A.; Simakov S.; Vassilevski Y.; Volpert V.}  Methods of Blood Flow Modelling. // Math Mod Nat Phenom. ~---2016;~---Vol.~11.~---P.1-25.
    
    \bibitem{Sherwin2003}{\it Sherwin S.J., Formaggia L., Peiro J.}  MComputational modelling of 1D blood
flow with variable mechanical properties and its application to the simulation
of wave propagation in the human arterial system // Int J Numer Meth Fl. ~---2003;~---Vol.~43(6).~---P.673–700.

    \bibitem{Xiao2013}{\it Xiao N., Humphrey J.D., Figueroa C.A.}  Multi-scale computational model of
three-dimensional hemodynamic within a deformable full-body arterial network // J. Comput. Phys. ~---2013;~---Vol.~244.~---P.22-40.

    \bibitem{Yakhot2005}{\it Yakhot A., Grinberg L., Nikitin N.}  Modeling rough stenosis by an immersedboundary
method // Journal of Biomechanics. ~---2005;~---Vol.~38.~---P.1115–
1127.

    \bibitem{Crosetto2011}{\it Crosetto P., Reymond P., Deparis S. et al.}  Fluid-structure interaction simulation
of aortic blood flow // Computers and Fluids. ~---2011;~---Vol.~43(1).~---P.46–57

    \bibitem{Barker2010}{\it Barker A.T., Cai X.C.}  Scalable parallel methods for monolithic coupling in
fluid-structure interaction with application to blood flow modeling // Journal
of Computational Physics. ~---2010;~---Vol.~229(3).~---P.642–659

    \bibitem{Hassani2006}{\it Hassani K., Navidbakhsh M., Rostami M.}  Simulation of the cardiovascular
system using equivalent electronic system // Biomedical papers of the Medical
Faculty of the University Palacky, Olomouc, Czechoslovakia. ~---2006;~---Vol.~150(1).~---P.105–112
      
    \bibitem{Mynard2012}{\it Mynard J.P., Davidson M.R., Penny D.J., Smolich J.J.} A simple, versatile
valve model for use in lumped parameter and one-dimensional cardiovascular
models // International Journal for Numerical Methods in Biomedical Engineering. ~---2012;~---Vol.~28(6).~---P.626–641

    \bibitem{Carlo2003}{\it A. Carlo Di, P. Nardinocchi, G. Pontrelli, L. Teresi} A heterogeneous approach for modelling blood flow in an arterial segment // Simulation in
Biomedicine. ~---2003;~---Vol.~5.~---P.69–78

    \bibitem{Hill1910}{\it Hill, A. V.} The possible effects of the aggregation of
the molecules of haemoglobin on its dissociation curves // J. Physiol. ~---1910;

    \bibitem{Adair1925}{\it Adair G. S.} The hemoglobin system. VI. The oxygen
dissociation curve of hemoglobin // J. Biol. Chem. ~---1925;~---Vol.~63.~---P.529–545

    \bibitem{Winslow1983}{\it Winslow R.M., Samaja N. J,  Rossi-
Bernardi L.,  Shrager R. I.} Simulation of continuous
blood O2 equilibrium over physiological pH, DPG, and
PCO2 range // J. Appl. Physiol.: Respirat. Environ. Exercise
Physiol. ~---1983;~---Vol.~54.~---P.524–529

    \bibitem{Kelman1966}{\it Kelman G. R.} Digital computer subroutine for the conversion
of oxygen tension into saturation // J. Appl. Physiol. ~---1966;~---Vol.~21.~---P.1375–1376

    \bibitem{Huang1994}{\it Huang N. S., Hellums J. D.} A theoretical model for
gas transport and acid/base regulation by blood flowing in
microvessels // Microvasc. Res. ~---1994;~---Vol.~48.~---P.364–388

    \bibitem{Dash2010}{\it Dash R., Bassingthwaighte J. B.} Erratum to: Blood HbO2 and HbCO2
dissociation curves at varied O2, CO2, pH, 2,3-
DPG and temperature levels // Annals of Biomedical Engineering. ~---2010;~---Vol.~38(4).~---P.1683–1701

 \bibitem{Magomedov1988} {\it Magomedov K.M., Kholodov A.S.} Grid-characteristic numerical methods, ~---Moscow:  Nauka, 1988.
 
 \bibitem{Danilov2016}{\it Danilov A., Ivanov Y., Pryamonosov R., Vassilevski Y.} Methods of graph network reconstruction in personalized medicine // Int J Num Meth Biomed Eng. ~---2016;~---Vol.~32.~---P.1–20
 
 \bibitem{Pudney1998}{\it Pudney C.} Distance-ordered homotopic thinning: a skeletonization algorithm for 3D digital images // Comput. Vis. Image Underst. ~---1998;~---Vol.~72.~---P.404--413
 
 \bibitem{Mead1961}{\it Mead J.} Mechanical properties of lungs ~---1961;~---Vol.~41.~---P.281-330

 \bibitem{bel2005} {\it Белоцерковский З.Б.} Эргометрические и кардиологические критерии физической работоспособности у спортсменов,~---Москва:  Советский спорт, 2005

 \bibitem{Burnley2002}{\it	Burnley M., Doust J.H., Ball D., Jones A.M.} Effects of prior heavy exercise on VCO2 kinetics during heavy exercise are related to changes in muscle activity // Journal of Applied Physiology. ~---2002;~---Vol.~93(1).~---P.167-174
 
 \bibitem{huber1984} {\it Хьюбер П.} Робастность в статистике,~---Москва:  Мир, 1984
  
 \bibitem{Boggs1988}{\it	Boggs P.T., Spiegelman C.H., Donaldson  J.R.} A Computational Examination of Orthogonal Distance Regression // Journal of Econometrics. ~---1988;~---Vol.~38(1-2).~---P.169-201
  
 \bibitem{efron1988} {\it Эфрон Б.} Нетрадиционные методы многомерного статистического анализа,~---Москва:  Финансы и статистика, 1988
  
  \bibitem{Wales1997}{\it Wales D.J., Doye J.P.K.} Global Optimization by Basin-Hopping and the Lowest Energy Structures of Lennard-Jones Clusters Containing up to 110 Atoms // Journal of Physical Chemistry. ~---1997;~---Vol.~101.~---P.5111
  
  \bibitem{Storn1997}{\it Storn R., Price K.} Differential Evolution - a Simple and Efficient Heuristic for Global Optimization over Continuous Spaces // Journal of Global Optimization. ~---1997;~---Vol.~11.~---P.341 - 359

  \bibitem{Wijdicks2007}{\it Wijdicks E.F.} Biot’s breathing // J Neurol Neurosurg Psychiatry. ~---2007;~---Vol.~78(5).~---P.512-513
    
  \bibitem{Pearce2002}{\it Pearce J.M.S.} Cheyne-Stokes respiration // J Neurol Neurosurg Psychiatry. ~---2002;~---Vol.~72.~---P.595
  
  \bibitem{Colice2004}{\it Colice G.L.} Categorizing Asthma Severity: An Overview of National Guidelines // Clinical Medicine \& Research. ~---2004;~---Vol.~2(3).~---P.155-164
  
  \bibitem{Defares1964} {\it Defares J. G.} Principles of feedback control and their application to the respriatory control system in Handbook of Physiology,~---Washington, D.C:  Am. Physiol. Soc., 1964
  
  \bibitem{Khoo1982}{\it Khoo M.C., Kronauer R.E., Strohl K.P., Slutsky A.S.} Factors inducing periodic breathing in humans: A general model // Journal of applied physiology: respiratory, environmental and exercise physiology. ~---1982;~---Vol.~53(3).~---P.644-659
  
  \bibitem{Cabrera1999}{\it Cabrera M.E., Saidel G.M., Kalhan S.C.} Lactate metabolism during exercise: analysis by an integrative systems model // American journal of physiology. Regulatory, integrative and comparative physiology. ~---1999;~---Vol.~277.~---P.R1522-R1536
  
  \bibitem{Maciejewski20013}{\it Maciejewski H., Bourdin M., Lacour J. R., Denis C., Moyen B.and Messonnier L.} Lactate accumulation in response to supramaximal exercise in rowers // Scandinavian Journal of Medicine \& Science in Sports. ~---2013;~---Vol.~23(5).~---P.585-592
  
  \bibitem{Hubbard1973}{\it Hubbard J.L.} The effect of exercise on lactate metabolism // The journal of physiology. ~---1973;~---Vol.~231(1).~---P.1-18
  
  \bibitem{Lai2009}{\it Lai N., Zhou H., Saidel G.M.,  Wolf M.,  McCully K., Gladden L.B., 
Cabrera M.E.} Modeling oxygenation in venous blood and skeletal muscle in response
to exercise using near-infrared spectroscopy // J Appl Physiol. ~---2009;~---Vol.~106.~---P.1858–1874
  
\bibitem{Hairer1999} {\it Хайрер Э. Ваннер Г.} Решение обыкновенных дифференциальных уравнений: жесткие и дифференциально-алгебраические задачи,~---Москва:  Мир, 1999  

\end{thebibliography}
