\chapter*{Заключение}						% Заголовок
\addcontentsline{toc}{chapter}{Заключение}	% Добавляем его в оглавление

В рамках данной работы были разработаны вычислительные методы расчета глобального транспорта дыхательных газов в организме человека при изменении условий внешней и внутренней среды.
В ходе достижения поставленной цели был решён ряд задач:

\begin{itemize}
\item
Разработан вычислительный метод расчета альвеолярного газообмена с использованием 0D-1D математической модели на основе персонифицированных CT-данных пациента  
\item
Выполнено тестирование вычислительного метода применительно к поиску оптимальных параметров искусственной вентиляции легких
\item
Выполнено численное исследование изменений рисунков дыхания и эффективности альвеолярного газообмена при наличии различных патологий (периодическое и кластерное дыхание, астма)
\item
Разработан алгоритм определения аэробного и анаэробного порогов по нескольким физиологическим показателям
\item
Разработана комплексная модель глобального транспорта в организме при физической нагрузке
\item
Разработана методика идентификации параметров модели по экспериментальным данным тестов с возрастающей нагрузкой.
\end{itemize}


\clearpage