\chapter*{Заключение}						% Заголовок
\addcontentsline{toc}{chapter}{Заключение}	% Добавляем его в оглавление

В рамках данной работы были разработаны вычислительные методы расчета глобального транспорта дыхательных газов для исследования состояния организма человека при наличии патологий дыхательных путей (например периодическое и кластерное дыхание, астма), изменении условий внешней среды (например гиперкапния, искусственная вентиляции легких, физическая нагрузка).

В ходе достижения поставленной цели был решён ряд задач:

\begin{itemize}
\item
Разработан вычислительный метод и программный комплекс расчета альвеолярного газообмена с использованием 0D-1D математической модели и данных компьютерной томографии легких пациента  
\item
Выполнен поиск оптимальных для пациента параметров искусственной вентиляции легких
\item
Выполнено численное исследование изменений рисунков дыхания и эффективности альвеолярного газообмена при наличии различных патологий (периодическое и кластерное дыхание, астма)
\item
Разработан метод и программный комплекс для определения аэробного и анаэробного порогов
\item
Разработана модель глобального транспорта газов в организме при физической нагрузке
\item
Разработан метод и программный комплекс для идентификации параметров модели по экспериментальным данным тестов с возрастающей нагрузкой.
\end{itemize}

\clearpage